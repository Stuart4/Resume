%%%%%%%%%%%%%%%%%%%%%%%%%%%%%%%%%%%%%%%%%
% Friggeri Resume/CV
% XeLaTeX Template
% Version 1.2 (3/5/15)
%
% This template has been downloaded from:
% http://www.LaTeXTemplates.com
%
% Original author:
% Adrien Friggeri (adrien@friggeri.net)
% https://github.com/afriggeri/CV
%
% License:
% CC BY-NC-SA 3.0 (http://creativecommons.org/licenses/by-nc-sa/3.0/)
%
% Important notes:
% This template needs to be compiled with XeLaTeX and the bibliography, if used,
% needs to be compiled with biber rather than bibtex.
%
%%%%%%%%%%%%%%%%%%%%%%%%%%%%%%%%%%%%%%%%%

\documentclass[print]{friggeri-cv} % Add 'print' as an option into the square bracket to remove colors from this template for printing


\begin{document}

\header{Jacob}{Stuart}{Software Engineering Intern} % Your name and current job title/field

%----------------------------------------------------------------------------------------
%	SIDEBAR SECTION
%----------------------------------------------------------------------------------------

\begin{aside} % In the aside, each new line forces a line break
\section{Contact}
14003 Waterford Ln
San Diego, CA 92129
USA
~
(858) 602-2088
~
\href{mailto:stuart4@purdue.edu}{Stuart4@Purdue.edu}
\href{http://stuartresearch.org}{StuartResearch.org}
\href{https://www.github.com/Stuart4}{Github://Stuart4}
~
\section{Technical Skills}
Java, C/C++,
Android Development,
Linux Kernel,
Git Source Control,
SQL, Node.js
\end{aside}

%----------------------------------------------------------------------------------------
%	EDUCATION SECTION
%----------------------------------------------------------------------------------------

\section{Education}

\begin{entrylist}

%------------------------------------------------

\entry
{2014--Now}
{Bachelor of Science - Computer Science {\normalfont (3.6/4.0)}}
{Purdue University}
{Minor in Management
	\begin{itemize}
	\item Expected May 2017
	\item Dean's List
	\end{itemize}}

%------------------------------------------------

\end{entrylist}

\section{Experience}

\begin{entrylist}

\entry
{2015--Now}
{Scientific Applications Developer}
{Rosen Center for Advanced Computing}
{Architecting tools used in the maintenance and operation of Purdue's High\\ Performance Computing resources.
\begin{itemize}
	\item Software Package Maintenance
	\item Kernel Optimization
	\item Quality Assurance and Trouble-Shooting
\end{itemize}}

\entry
{2015--Now}
{Undergraduate TA}
{CS190 (Tools and Resources)}
{Instructing students on advanced techniques regarding shells, advanced\\ editors, source control, and debugging.}


\end{entrylist}

%----------------------------------------------------------------------------------------
%	WORK EXPERIENCE SECTION
%----------------------------------------------------------------------------------------

\section{Projects}

\begin{entrylist}
%------------------------------------------------

\entry
{Ongoing}
{Linux Kernel Development}
{Kernel C}
{Designing and integrating various modules and drivers into the Linux Kernel and utilized advanced Kernel compilation techniques.
\begin{itemize}
\item Investigating the sending of git patches in plain text
\item Generating pseudo Linux Kernel Mailing List interactions
\end{itemize}}

%------------------------------------------------

\entry
{Ongoing}
{Tapzu}
{Android Application}
{Formulating advanced content aggregation front end. Exploiting advanced\\ Android patterns such as MVVM and databinding.
\begin{itemize}
\item Devised asynchronous fetchings in RxJava
\item Cultivated 500 active users on Google Play
\end{itemize}}

%------------------------------------------------

\entry
{May 2015}
{Radio 91x}
{Android Application}
{Reverse engineered a closed streaming API into an Android application\\ designed to stream music in the background with song history and favorites.
\begin{itemize}
\item Constructed foreground service with IBinder communication
\item Devised SQLITE backend
\end{itemize}}


%------------------------------------------------

\entry
{June 2015}
{Push Torrent}
{Node.js Server Applicaton}
{Developed nonblocking server application to observe a websocket and\\ intelligently download links.
\begin{itemize}
\item Implements websockets
\item Incorporated Gzip support
\end{itemize}}

%------------------------------------------------

\entry
{April 2015}
{IRC Client}
{C++ with GTK GUI}
{Reimplemented server and client of the IRC specification.
\begin{itemize}
\item Integrated GTK GUI
\item Wrote featureful C Networking
\end{itemize}}

\end{entrylist}

%----------------------------------------------------------------------------------------
%	AWARDS SECTION
%----------------------------------------------------------------------------------------

\end{document}
